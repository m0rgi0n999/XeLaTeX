\documentclass[../main.tex]{subfiles}

\begin{document}

\noindent{Each monster description is organised in the same general format, as outlined below. For complete information about the characteristics of monster, consult Chapter 7: Glossary (starting on page 305), the \textit{Player's Handbook}, or the \textit{Dungeon Master's Guide}.}

\dndsection{STATISTICS BLOCK}
\noindent{This portion of a monster description contains basic game information on the creature.}

\dndsubsection{Name}
\noindent{This is the name by which the creature is generally known. The descriptive text may provide other names.}

\dndsubsection{Size and Type}
\noindent{This line describes the creature's size (Huge, for example). Size categories are defined in the Glossary. A size modifier applies to the creature's Armour Class (AC) and attack bonus, as well as to certain skills. A creature's size also determines how far it can reach to make a melee attack and how much space it occupies in a fight (see Space/Reach. below).}\\
\indent{The size and type line continues with the creature's type (giant, for example). Type determines how magic affects a creature for example, the \textit{hold animal} spell affects only creatures of the animal type. Type determines certain features, such as Hit Dice size, base attack bonus, base saving throw bonuses, and skill points. For quick reference, the Glossary gives a full description of the features and traits of each type and subtype.}

\dndsubsection{Hit Dice}
\noindent{This line gives the creature's number and type of Hit Dice, and lists any bonus hit points. A parenthetical note gives the average hit points for a creature of the indicated number of Hit Dice.}\\
\indent{A creature's Hit Dice total is also treated as its level for determining how spells affect the creature, its rate of natural healing, and its maximum ranks in a skill.}

\dndsubsection{Initiative}
\noindent{This line gives the creature's modifier on initiative checks}

\dndsubsection{Speed}
\noindent{This line gives the creature's tactical speed on land (the amount of distance it can cover in one move action). If the creature wears armour that reduces speed, the creature's base land speed follows.}
\indent{If the creature has other modes of movement, these are given after (or in place of) the land speed. Unless noted otherwise, modes of movement are natural (not magical). See the Glossary for information on movement modes.}

\dndsubsection{Armour Class}
\noindent{The Armour Class line gives the creature's AC for normal combat and includes a parenthetical mention of the modifiers contributing to it (usually size, Dexterity, and natural armour). The creature's touch and flat-footed ACs follow the combat-ready AC.}
\indent{A creature's armour proficiencies (if it has any) depend on its type, but general a creature is automatically proficient with any kind of armour it is described wearing (light, medium, or heavy), and with all lighter kinds of armour.}

\dndsubsection{Base Attack/Grapple}
\noindent{The number before the slash on this line is the creature's base attack bonus (before any modifiers are applies). The DM usually won't need this number, but it can be handy sometimes, especially if the creature has the Power Attack or Combat Expertise feats.}\\
\indent{The number after the slash is the creature's grappling bonus, which is used when the creature makes a grapple attack or when someone tries to grapple the creature. The grapple bonus includes all modifiers that apply to the creature's grapple checks (base attack bonus, Strength modifier, special size modifier, and any other applicable modifier, such as a racial bonus on grapple checks).}

\dndsubsection{Attack}
\indent{This line shows the single attack the creature makes with an attack action. In most cases, this is also the attack the creature uses when making an attack of opportunity as well. The attack line provides the weapon used (natural or manufactures), attack bonus, and form of attack (melee or ranged). The attack bonus given includes modification for size and Strength (for melee attacks) or Dexterity (for ranged attacks). A creature with the Weapon Finesse feat can use its Dexterity modifier on melee attacks.}\\
\indent{If the creature uses natural attacks, the natural weapon given here is the creature's primary natural weapon (see Glossary).}\\
\indent{If the creature has several weapons at its disposal, the alternatives are shown, with each different attack separated by the word "or".}\\
\indent{A creature can use one of its secondary natural weapons (see the Glossary) when making an attack action, but if it does it takes an attack penalty, as noted in the Full Attack section below.}\\
\indent{The damage that each attack deals is noted parenthetically. Damage from an attack is always at least 1 point, even if a subtraction from a die roll reduces the result to 0 or lower.}

\dndsubsection{Full Attack}
\noindent{This line shows all the physical attacks the creature makes when it uses a full-round action to make a full attack. It gives the number of attacks along with the weapon, attack bonus, and form of attack (melee or ranged). The first entry is for the creature's primary weapon, with an attack bonus including modifications for size and Strength (for melee attacks) or Dexterity (for ranged attacks). A creature with the Weapon Finesse feat can use its Dexterity modifier on melee attacks.}\\
\indent{The remaining weapons are secondary, and attacks with them are made with a -5 penalty to attack roll, no matter how many there are. Creatures with the Multiattack feat (see page 304) take only a -2 penalty on secondary attacks.}\\
\indent{The damage that each attack deals is noted parenthetically. Damage from attacks is always at least 1 point, even if a subtraction from a die roll reduces the result to 0 or lower.}\\
\indent{A creature's primary attack damage includes its full Strength modifier (1-\textonehalf\ times its Strength bonus if the attack is with the creature's sole natural weapon) and is given first. Secondary attacks add only \textonehalf\ the creature's Strength bonus and are given second in the parentheses.}\\
\indent{If any attacks also have some special effect other than damage (poison, disease, energy drain, and so forth), that information is given here.}\\
\indent{Unless noted otherwise, creatures using natural weapons deal double damage on critical hits.}\\
\indent{\textbf{Manufactured Weapons:} Creatures that use swords, bows, spears, and the like follow the same rules as characters do. The bonus for attacks with two-handed weapons is 1-\textonehalf\ times the creature's Strength modifier (if it is a bonus), and is given first. Off-hand weapons add only \textonehalf\ the Strength bonus and are given second in the parentheses.}

\dndsubsection{Space/Reach}
\noindent{This line describes how much space the creature takes up on the battle grid and thereby needs to fight effectively, as well as how close it has to be to threaten an opponent. The number before the slash is the creature's space, or how many feet one side of the creature occupies (refer to the \textit{Dungeon Master's Guide} for additional details). For example, a creature with a space of 15 feet occupies a 3-square-by-3-square space on the battle grid. The number after the slash is the creature's natural reach. If the creature has exceptional reach due to a weapon, tentacle, or the like, the extended reach and its source are noted in parentheses at the end of the line}

\dndsubsection{Special Attacks and Special Qualities}
\noindent{Many creatures have unusual abilities, which can include special attack forms, resistance or vulnerability to certain types of damage, and enhanced senses, among others. A monster entry breaks these abilities into special attacks and special qualities. The latter category includes defences, vulnerabilities, and other special abilities that are not modes of attack. A special ability is either extraordinary (Ex), spell-like (Sp), or supernatural (Su). See the Glossary for definitions of special abilities. Additional information (when needed) is provided in the creature's descriptive text.}\\
\indent{When a special ability allows a saving throw, the kind of save and the save DC is noted in the descriptive text. Most saving throws against special abilities have DCs calculated as follows: 10 + \textonehalf the attacker's racial Hit Dice + the relevant ability modifier. The save DC is given in the creature's description along with the ability on which the DC is based.}

\dndsubsection{Saves}
\noindent{This line gives the creature's Fortitude, Reflex, and Will save modifiers.}

\dndsubsection{Abilities}
\noindent{This line lists the creature's ability scores, in the customary order: Str, Dex, Con, Int, Wis, Cha. Except where noted otherwise, each creature is assumed to have the standard array of ability scores before racial adjustments (all 11s and 10s). To determine any creature's racial ability adjustments, subtract 10 from any even-numbered ability score and subtract 11 from any odd-numbered score. (Exceptions are noted in the Combat section of a creature's descriptive text.) Humanoid warriors are generally built using the nonelite array: 13, 12, 11, 10, 9, 8. Advanced creatures (such as the hound archon hero) are built using the elite array: 15, 14, 13, 12, 10, 8.}\\
\indent{Most abilities work as described in Chapter 1 of the \textit{Player's Handbook}, with exceptions given below.}\\
\indent{\textbf{Strength:} As noted on page 162 of the \textit{Player's Handbook}, quadrupeds can carry heavier loads than bipeds can. Any creature with four or more motive limbs can carry loads as a quadruped, even if it does not necessarily use all the limbs at once. For example, dragons carry loads as quadrupeds.}\\
\indent{\textbf{Intelligence:} A creature can speak all the languages mentioned in its description, plus one additional language per point of Intelligence bonus. Any creature with an Intelligence score of 3 or higher understands at least on language (Common, unless noted otherwise).}\\
\indent{\textbf{Nonabilities:} Some creatures lack certain ability scores. These creatures do not have an ability score of 0--they lack the ability altogether. The modifier for nonability is +0. Other effects of nonabilities are detailed in the Glossary.}

\dndsubsection{Skills}
\noindent{This line gives the creature's skills, along with each skill's modifier (including adjustments for ability scores, armour check penalties, and any bonuses from feats or racial traits). All listed skills are class skills, unless the creature has a character class (noted in the entry). A creature's type and Intelligence score determine the number of skill points it has.}\\
\indent{The skills section of the creature's description recaps racial bonuses and other adjustments to skill modifiers for the sake of clarity; these bonuses should not be added to the listed skill modifiers. An asterisk (*) besides the relevant score and in the Skills section of the descriptive text indicates a conditional adjustment, one that applies only in certain situations (for instance, a gargoyle gets an additional +8 bonus on Hide checks when it is concealed against a background of worked stone).}\\
\indent{\textbf{Natural Tendencies:} Some creatures simply aren't made for certain types of physical activities. Elephants, despite their great Strength scores, are terrible at jumping Giant crocodiles, despite their high Strength scores, don't climb well. Horses can't walk tightropes. If it seems clear to you that a particular creature simply is not made for a particular physical activity, you can say that the creates takes a -8 penalty on skill checks that defy its natural tendencies. In extreme circumstances (a porpoise attempting a Climb check, for instance) you can rule that the creature fails the check automatically.}

\dndsubsection{Feats}
\noindent{The line gives the creature's feats. A monster gains feats just as a character does--one for its first Hit Dice, a second feat if it has at least 3 HD, and an additional feat for every additional 3 HD. (For example, a 9 HD creature is entitled to four feats.)}\\
\indent{Sometimes a creature has one or more bonus feats, marked with a superscript B (\textsuperscript{B}). Creatures often do not have the prerequisites for a bonus feat. If this is so, the creature can still use the feat. If you wish to customise the creature with new feats, you can reassign its other feats, but not its bonus feats. A creature cannot have a feat that is not a bonus feat unless it has the feat's prerequisites.}

\dndsubsection{Environment}
\noindent{This line gives a type of climate and terrain where the creature is typically found. This describes a tendency, but is not exclusionary. A great wyrm gold dragon, for instance, has an environment entry of warm plains, but could also be encountered underground, in cold hills, or even on another plane of existence. See Chapter 3 of the \textit{Dungeon Master's Guide} for more on terrain types and climate.}

\dndsubsection{Organisation}
\noindent{This line describes the kinds of groups the creature might form. A range of numbers in parentheses indicates how many combat-ready adults are in each type of group. Many groups also have a number of noncombatants, expressed as a percentage of the fighting population. Noncombatants can include young, the infirm, slaves, or other individuals who are not inclined to fight. A creature's Society section may include more details on noncombatants.}\\
\indent{If the organisation line contains the term "domesticated," the creature is generally found only in company of other creatures, whom it servers in some capacity.}

\dndsubsection{Challenge Rating}
\noindent{This shows the average level of a party of adventurers for which one creature would make an encounter of moderate difficulty. Assume a party of four fresh characters (full hit points, full spells, and equipment appropriate to their levels). Given reasonable luck, the party should be able to win the encounter with some damage but no casualties. For more information about Challenge Ratings, see pages 36 and 48 of the \textit{Dungeon Master's Guide}.}

\dndsubsection{Treasure}
\noindent{This line reflects how much wealth the creature owns and refers to Table 3-5: Treasure on page 52 of the \textit{Dungeon Master's Guide}. In most cases, a creature keeps valuables in its home or lair and has no treasure with it when it travels. Intelligent creatures that own useful, portable treasure (such as magic items) tend to carry and use these, leaving bulky items at home. See the Glossary for more details on using the Treasure line of each monster entry.}

\dndsubsection{Alignment}
\noindent{This line gives the alignment that the creature is most likely to have. Every entry includes a qualifier that indicates how broadly that alignment applies to the species as a whole. See the Glossary for details.}

\dndsubsection{Advancement}
\noindent{This book usually describes only the most commonly encountered version of a creature (though some entries for advanced monsters can be found). The advancement line shows how tough a creature can get, in terms of extra Hit Dice. (This is not an absolute limit, but exceptions are extremely rare.) Often, intelligent creatures advance by gaining a level in a character class instead of just gaining a new Hit Die.}

\dndsubsection{Level Adjustment}
\noindent{This line is included in the entries for creatures suitable for use as player characters or as cohorts (usually creatures with Intelligence scores of at least 3 and possessing opposable thumbs). Add this number to the creature's total Hit Dice, including class levels, to get the creature;s effective character level (ECL). A character's ECL affects the experience the character earns, the amount of experience the character must have before gaining a new level, and the character's starting equipment. See pager 172, 199, and 209 of the \textit{Dungeon Master's Guide} for more information.}

\dndsection{DESCRIPTIVE TEXT}
\noindent{The body of each entry opens with a sentence or two that describes what the player characters might see on first encountering a monster, followed by a short description of the creature: what it does, what it looks like, and what is most noteworthy about it. Special sections describe how the creature fights and give details on special attacks, special qualities, and feats.}

\end{document}